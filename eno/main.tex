\documentclass[MathSerif]{beamer}

\usetheme{UiB}

\usepackage{array}
\usepackage{graphicx}

\title[CFD-note] % (optional, only for long titles)
{The example title of Beamer}
\subtitle{with a little subtitle}
\author[Author, Anders] % (optional, for multiple authors)
{F.~Author\inst{1} \and S.~Anders\inst{2}}
\institute[] % (optional)
{
  \inst{1}%
  Institute of Computer Science\\
  University Here
  \and
  \inst{2}%
  Institute of Theoretical Philosophy\\
  University There
}
\date[Date 2004] % (optional)
{CFD-note, 2020}

\subject{Computational Fliud Dynamics}


\begin{document}


\section[First]{The First Section}
% table of content =============================
\begin{frame}
	\frametitle{Table of Contents}
	\tableofcontents[]
\end{frame}

\begin{frame}
    \frametitle{Mathematics}

    \begin{theorem}[Fermat's little theorem]
        For a prime~\(p\) and \(a \in \mathbb{Z}\) it holds that \(a^p \equiv a \pmod{p}\).
    \end{theorem}

    \begin{proof}
        The invertible elements in a field form a group under multiplication.
        In particular, the elements
        \begin{equation*}
            1, 2, \ldots, p - 1 \in \mathbb{Z}_p
        \end{equation*}
        form a group under multiplication modulo~\(p\).
        This is a group of order \(p - 1\).
        For \(a \in \mathbb{Z}_p\) and \(a \neq 0\) we thus get \(a^{p-1} = 1 \in \mathbb{Z}_p\).
        The claim follows.
    \end{proof}
\end{frame}


% a normal slide ===============================
\begin{frame}

\frametitle{This is the first slide}
\begin{enumerate}
\item First item
\begin{itemize}
\item First subitem
\item Second subitem
\end{itemize}
\item Second item
\item Third item
\end{enumerate}
\[ y = \sum {i \in A} a i x i \]
\end{frame}
  

  
\section[Second]{The Second Section}
\begin{frame}

   \begin{block}{This is a Block}
      This is important information
   \end{block}

   \begin{alertblock}{This is an Alert block}
   This is an important alert
   \end{alertblock}

   \begin{exampleblock}{This is an Example block}
   This is an example 
   \end{exampleblock}

\end{frame}
  
\begin{frame}{Example of columns 1}
    \begin{columns}[T] % the "c" option specifies center vertical alignment
    \column{.5\textwidth} % column designated by a command
     There are two handy environments for structuring a slide: "blocks", which divide the slide (horizontally) into headed sections, and "columns" which divides a slide (vertically) into columns. Blocks and columns can be used inside each other.
    \column{.5\textwidth}
     There are two handy environments for structuring a slide: "blocks", which divide the slide (horizontally) into headed sections, and "columns" which divides a slide (vertically) into columns. Blocks and columns can be used inside each other. \\ into two lines
    \end{columns}
\end{frame}
% etc
\subsection{Example}

\begin{frame}
    \frametitle{Mathematics}

    \begin{example}
        The function \(\phi \colon \mathbb{R} \to \mathbb{R}\) given by \(\phi(x) = 2x\) is continuous at the point \(x = \alpha\),
        because if \(\epsilon > 0\) and \(x \in \mathbb{R}\) is such that \(\lvert x - \alpha \rvert < \delta = \frac{\epsilon}{2}\),
        then
        \begin{equation*}
            \lvert \phi(x) - \phi(\alpha)\rvert = 2\lvert x - \alpha \rvert < 2\delta = \epsilon.
        \end{equation*}
    \end{example}
\end{frame}


\end{document}