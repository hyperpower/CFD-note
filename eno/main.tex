\documentclass[MathSerif]{beamer}

\usetheme{UiB}

\usepackage{array}
\usepackage{graphicx}
\usepackage{graphics}

\title[CFD-note] % (optional, only for long titles)
{Essentially Non-Osicillatory Schemes}
\subtitle{simple guide}
\author[Chengsi ZHOU] % (optional, for multiple authors)
{Chengsi ZHOU}
%\institute[] % (optional)
%{ 
%  \inst{1}%
%  Institute of Computer Science\\
%  University Here
%  \and
%  \inst{2}%
%  Institute of Theoretical %Philosophy\\
%  University There
%}
\date[Feb 2020] % (optional)
{CFD-note, 2020}

%\subject{Computational Fliud %Dynamics}


\begin{document}

\section[Definition]{One Space Dimension}
% table of content =============================
\begin{frame}
	\frametitle{Table of Contents}
	\tableofcontents[]
\end{frame}

\begin{frame}
    \frametitle{Definition of One Space Dimension}
	Given a grid 
	\[ a = x_{\frac{1}{2}} < x_{\frac{3}{2}} < ... < x_{N-\frac{1}{2}} < x_{N + \frac{1}{2}} = b \]
\begin{center}
	\includegraphics[trim={0 5mm 0 5mm},clip]{./fig/one_dim_define.pdf}
\end{center}
	We define 
\begin{description}
	\item[Cells] \( I_i = [x_{i-\frac{1}{2}}, x_{i+\frac{1}{2}}]\)
	\item[Cell centers] \( x_i = \frac{1}{2} (x_{i-\frac{1}{2}} + x_{i+\frac{1}{2}})\)
	\item[Cell sizes] \( \Delta x = x_{i+\frac{1}{2}} - x_{i-\frac{1}{2}}\)
\end{description}
\end{frame}

\begin{frame}

\frametitle{One Dimensional Reconstruction}

\begin{columns}[T] % the "c" option specifies center vertical alignment
    \column{.4\textwidth} % column designated by a command
    Given the cell averages of a function \( v(x) \) 
    \[ \bar{v_i} \equiv \frac{1}{\Delta x_i}  \int_{x_{i-\frac{1}{2}}}^{x_{i+\frac{1}{2}}} v(\xi) d \xi\]
    find a polynomial $p_i(x)$, of degree at most $k-1$, for each cell $I_i$. It is a $k$-th order accurate approximation to $v(x)$
    \[p_i(x) = v(x) + O(\Delta x^k)\]
    \[x \in I_i\]
    \[i=1,...,N\]
    \column{.6\textwidth}
    \includegraphics[trim={0 0mm 0 0mm},clip]{./fig/vx_define.pdf}
    \end{columns}

\end{frame}


\begin{frame}

\frametitle{One Dimensional Reconstruction}

$p_i(x)$ gives approximations to the function $v(x)$ at the cell boundaries:

\[ v^{-}_{i+\frac{1}{2}} \leftarrow p_i(x_{i+\frac{1}{2}}) = v(x_{i+\frac{1}{2}}) + O(\Delta x^k)\]

\[ v^{+}_{i-\frac{1}{2}} \leftarrow p_i(x_{i-\frac{1}{2}}) = v(x_{i-\frac{1}{2}}) + O(\Delta x^k)\]

\[ i = 1, ..., N\]
\begin{center}
\includegraphics[trim={0 0mm 0 2mm},clip]{./fig/vfmp_define.pdf}
\end{center}

\end{frame}
  

  
\section[Second]{The Second Section}
\begin{frame}

   \begin{block}{This is a Block}
      This is important information
   \end{block}

   \begin{alertblock}{This is an Alert block}
   This is an important alert
   \end{alertblock}

   \begin{exampleblock}{This is an Example block}
   This is an example 
   \end{exampleblock}

\end{frame}
  
\begin{frame}{Example of columns 1}
    \begin{columns}[T] % the "c" option specifies center vertical alignment
    \column{.5\textwidth} % column designated by a command
     There are two handy environments for structuring a slide: "blocks", which divide the slide (horizontally) into headed sections, and "columns" which divides a slide (vertically) into columns. Blocks and columns can be used inside each other.
    \column{.5\textwidth}
     There are two handy environments for structuring a slide: "blocks", which divide the slide (horizontally) into headed sections, and "columns" which divides a slide (vertically) into columns. Blocks and columns can be used inside each other. \\ into two lines
    \end{columns}
\end{frame}
% etc
\subsection{Example}

\begin{frame}
    \frametitle{Mathematics}

    \begin{example}
        The function \(\phi \colon \mathbb{R} \to \mathbb{R}\) given by \(\phi(x) = 2x\) is continuous at the point \(x = \alpha\),
        because if \(\epsilon > 0\) and \(x \in \mathbb{R}\) is such that \(\lvert x - \alpha \rvert < \delta = \frac{\epsilon}{2}\),
        then
        \begin{equation*}
            \lvert \phi(x) - \phi(\alpha)\rvert = 2\lvert x - \alpha \rvert < 2\delta = \epsilon.
        \end{equation*}
    \end{example}
\end{frame}


\end{document}